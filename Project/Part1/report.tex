\documentclass[11pt]{article}

\usepackage{sectsty}
\usepackage{graphicx}
\usepackage{float}

% Margins
\topmargin=-0.45in
\evensidemargin=0in
\oddsidemargin=0in
\textwidth=6.5in
\textheight=9.0in
\headsep=0.25in

\title{CS 3600 Project Phase 1}
\author{Jake Gendreau, Michael Crapse}
\date{\today}

\begin{document}
\maketitle	

\tableofcontents
\newpage

\section{Project Plan}
    StockUp is a social and financial app which is a combination of fantasy football and real-world investment.
    On StockUp, users will be able to join a team, invest in stocks, and earn money for their team.
    \begin{itemize}
        \item Create an account with a starting balance of \$10,000.
        \item Join a team with other users.
        \item Invest those \$10,000 as they see fit.
        \item View current, historic, and predicted stock prices.
        \item Use the social tools to view their team members, user portfolios, and top performing investors.
    \end{itemize}

\section{Introduction}
    \subsection{Understanding of the Project}
        We understand the project to be something which requires a database, frontend, and backend.
        Additionally, we understand the project to require a "wow-factor" element. Additionally, the project
        must include a graph database and a set of queries regarding the graph database.

    \subsection{Interpretation of Requirements}
        The project needs to implement, on top of our relational database, either an emulation or actual
        implementation of a graph database, with the following queries:
        \begin{enumerate}
            \item \textbf{Co-Author Network} Our equivalent of a Co-Author network will be the user's teammates.
            A user will be able to type in their own username, or anyone else's username, and see what team
            they are on, and who else is on that team.
            \item \textbf{H-Index Filter} Our equivalent to an h-index filter will be sorting users based on
            their average portfolio performance. If they are making over a certain H-index amount, on average, per
            stock, then they will be listed. Essentially, this would be all of the highest performing users on the
            platform. It will list, under their name, what their top 5 stock values are.
            \item \textbf{Q1 Journal Influence Network} Our equivalent to a Q1 Journal Influence Network would be to
            list all of the teams which feature one or more users who have an H-index score greater than a 
            certain value.
        \end{enumerate}

\section{Technology Stack}
    \subsection{Backend}
        \subsubsection{Language} For our backend language, we will be utilizing Python. We chose python for a few reasons:
        \begin{enumerate}
            \item \textbf{Documentation and Support:} Python is an extremely well documented and supported language, and as a result, many StackOverflow
            forms, Reddit threads, or the actual documentation will have everything that we need to solve any problems we
            may face.
            \item \textbf{Libraries:} Extremely well supported libraries make implementing the features that we want to
            a much easier task than it would be if we had to implement them from scratch. Libraries such as Django
            or Flask for the web server, and PyTorch or XGBoost for machine learning provide extremely refined
            and well documented solutions for the features that we wish to add.
            \item \textbf{API Interfacing:} Since our project will be utilizing the Yahoo Finance API, we wanted to
            use a language that has easy support for it. Python has the yfinance library, which automatically calls
            the Yahoo Finance API and formats the results.
        \end{enumerate}

        \subsubsection{Libraries} We will be using several libraries in our project. Primarily, we will be
        using the following:
        \begin{enumerate}
            \item \textbf{Django:} We chose to use Django as our web server because of its "batteries included" approach,
            which means that many features that a website may want to use (e.g. login security, cookies,
            databases) have first-party developed modules which are well documented and easy to add.
            \item \textbf{XGBoost + PyTorch:} We chose to use XGBoost and PyTorch for the machine
            learning aspects of our project because they are extremely well documented, easy to use,
            and provide powerful tools for our application.
            \item \textbf{YFinance:} We chose to use YFinance instead of doing manual API calls because
            in testing, I was frequently limited by the API due to utilizing it too much. As a result,
            I had to find an alternative which wouldn't limit me (at least as much). YFinance is a 
            well-supported option which uses a reliable source for its information (Yahoo Finance).
        \end{enumerate}

    \subsection{Database}
        Due to our choice to use Python and Django, we were left with a few natively supported options for
        our database management software. After some conversation, we decided that SQLite3 would be the best
        option, as both of us have experience in the software, and the limited installation size and feature 
        set means that it is easier to install, use, and develop with.

    \subsection{Visualization Tools}
        We chose to use Chartjs for our database visualization due to its compatibility with Django,
        nice looking output, and documentation support.

\section{ER Diagram}
    \begin{figure}[H]
        \includegraphics[width=\linewidth]{./er_model.png}
        \caption{ER Model for StockUp}
        \label{fig:er_model}
    \end{figure}

\section{Timeline of Deliverables}
    \begin{itemize}
        \item \textbf{February 26, 2026:} The initial draft for the project must be ready, along with
        a presentation about the project.
        \item \textbf{March 17, 2026:} The project must have a finalized database design, all interfaces
        designed, all functions planned out, and at least one functional interface with backend. Additionally,
        the stock price predictor model must be functional at this time. It should be done with Phase II at this
        point.
        \item \textbf{April 15, 2026:} All parts of the project must work in isolation of each other. The website
        should be fully functional, the backend and API should be fully functional, and the database should be
        fully functional. At this point, the project should be done with Phase III.
        \item \textbf{May 1, 2026:} The project should be fully functional at this point. The required queries
        should be in place, and all aspects of the project should properly interact with minimal bugs. Phase IV
        should be complete at this point.
        \item \textbf{May 15, 2026:} The project must be complete by this point.
    \end{itemize}


\end{document}